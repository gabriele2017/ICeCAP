\documentclass[10pt,a4paper]{article}
\begin{document}
\hspace*{1cm} 
{\huge ICeCAP} 
\vspace*{2cm}

ICeCAP is written in ANSI-C and provides flexible, multi-scale analysis of Hi-C and Capture-C/Capture-Hi-C data. Dynamical, pointer based memory allocation and mapping of sub-genomic regions allow interaction maps to be resolved computationally at single restriction fragment, "meta-fragment" or uniform resolution.
The output files can be converted to $hic$ and $cool$ format with the $Juicer$ and $Hi-Glass$ tools.
ICeCAP has been developed at the Institute of Cancer Research and the Wellcome Centre for Human Genetics by Gabriele Migliorini.

GNU GPL v3 license applies.\\

\section*{Requirements:}
*******************\\
ICeCap runs in a Linux environment running bash shell with GNU environment default functions (gawk, sort, sed, etc...) available at run time.
Compilation is done using the GNU Compiler (GCC).  This may be changed by editing the make file (see below).
Read alignment is carried out using bowtie2, version 2.2.X.  This should be available in the PATH environtment variable.  To check the available version of bowtie type "bowtie2 --version" at the command line.
Bowtie reference files should be generated using the same reference genome as the one used during alignment (see the bowtie manual for details).
Bed formatted mapability files for the species under study should be placed in the subfolder "reference\_map". See the README file in that folder.
The mapability files should be generated by the same version of the aligner below. A script to help building mapability files is included.
Perl version v5.10.1 or later is expected, earlier versions may also work.
R version 3.X.X or later is expected.
Free disk space that is at least double the size of the input fastq files is recommended, more if intermediate files are kept (see options).
A considerable amount of RAM may also be needed depending on the number of fragments or bins being analysed.  Analysis of data from a typical experiment involving the human genome digested with HindIII may require around 40 gigabytes of RAM. A minimum 1kb resolution is the default setting, though higher resolution is in principle possible, when considering .e.g. four base cutter enzyme or other digestion protocols.

Please make sure all dependencies are met. E.g:

export PATH bowtie2/2.2.5/";
export PATH parallel/20161122/bin/";
export PATH R/3.4.3/lib64/R/bin/";

\section{Compilation:}
*******************\\
Unzip/untar the file ICeCap.tar.gz:

$>$ tar -zxvf ICeCap.tar.gz\\
Run the make file found in the ICeCap directory to compile the ICeCap binary file. The ICeCap directory can either be in your home folder or 
can be installed with administration privileges in the root directory.
Make sure you add the ICeCap folder to your PATH environment in this case.
In the ICeCap folder, at the command line prompt, type:\\

$>$ make\\
That is all. Test shell scripts for submitting ICeCAP on a suitable queuing environment are provided.

\vspace*{1cm}

\section{Running ICeCap:}
*******************\\
ICeCap takes unaligned sequencing data in the FASTQ format.  The data is processed in two stages: \\

*Firstly, sequencing reads are processed to take into account ligation junctions. Flashing occurs followed by insilico digestion. Reporter/baited fragments are aligned using bowtie2.  Valid ditags, along the lines of other NGS pipelines, e.g. HICCUP,  are identified and stored in SAM-like format files. Allele specific calibration of reads is available in the option list (see below).\\

*Secondly, a matrix of interacting intervals is populated. The resolution can be choosen to be fragment based or uniform-bin based. Iterative correction and interaction distance normalisation is carried out along the lines of the Sinkhorn-Knopp algorithm via e.g. cool tools.  A list of significant interactions based on the HICCUPs algorithm, namely a local Poisson backgroup approach, can also be produced using the hic files generated by the juicer tools.


\subsection{**STAGE 1: read alignment, pairing, deduplication, and ditag validation.}

\vspace*{1cm}
To process sequencing data from, e.g., a non-capture Hi-C experiment, using the HindIII restriction enzyme, run in your system, as a $single$ line command, e.g.:\\

$\bullet$ /path/to/binary/ICeCap -E AAGCTT -C 1 \\
-L /path/to/your/fastqfiles one strand/ \\
-R /path/to/your/fastqfiles one strand/ \\
-P /path/to/your/bowtie/reference/directory/ \\
-l sampleid \\
-T 100 \\
-f /path/to/your/ICeCap/directory/ \\
-D /path/to/your/outputsam/folder/ \\

To process capture Hi-C data a BED formated file listing the captured intervals needs to be provided using the -B option. \\

$\bullet$ ICeCap -B path/to/baits.bed -E AAGCTT -C 1  \\
-L /path/to/your/inputfastq/file(s)  \\
-R /path/to/your/inputfastq/file(s)  \\
-P /path/to/your/bowtie/reference/directory/ \\
-l sampleid \\
-T 100 \\
-f /path/to/your/ICeCap/directory/ \\
-D /path/to/your/outputsam/folder/ \\

ICeCap will carry out a virtual digestion of the reference genome provided (default hg19), generating a list of fragments.
Multiple files can be concatenated passing to the -L/-R flags a folder containing all files, e.g. both the left and right strand flag, e.g. $-L /pathR$,$-R /pathL$. The strings $1.fastq$ and $2.fastq$ are expected on each file, file names should sort match, and total number of reads on each strand should be equal.
Make sure you submit the command above with a suitable queuing environment, e.g. MOAB, LSF etc.

STAGE 1 will produced SAM files listing valid ditags to be used for subsequent analysis.  A header can be found in the "data" directory.
The following output files will be produced in the output directory, e.g. /sampleid/data (indicated with the -D option).

File name                                                         |  File contents\\
----------------------------------------------------------------------------------\\
sampleid/data/SIZES                                     |  Ditag sizes.\\
sampleid/data/GSIZE                                     |  Valid ditag sizes.\\
sampleid/data/BSIZE                                     |  Invalid ditag sizes.\\
sampleid/data/sampleid.png                            |  Plot of bona-fide and filtered di-tags size distributions \\
sampleid/data/Invalid\_ditag\_chart.pdf                 |  Pie chart with the relative ratio of invalid di-tags filtered out.
sampleid/data/sampleid.COUNTS.txt                     |  Statistics about the numbers of aligned reads and valid ditags.\\
sampleid/data/sampleid.pairs.chr*.sam.bfide           |  SAM-like file listing valid ditags.\\
sampleid/data/sampleid.pairs.chr*.sam.bfide.ontarget  |  SAM-like file listing valid ditags corresponding to enriched ditags in \\
                                                                     the case of capture Hi-C or all ditags in the case of non-capture Hi-C.\\

\subsection{**STAGE 2: Matrix allocation and interaction calling.}

To run this stage, use the -N option to allocate di-tags to memory and postprocessing analysis.
The -S option will run statistical analysis on the two pools of data obtained: enriched to non-enriched (E-N) as well as the enriched to enriched (E-E) interactions.
The argument to the -S option indicats the type of distribution that will be used as a local background model in the three-filtered, HiCCUPs like, method.  

Example of interaction analysis in capture Hi-C data carried out on chromosomes 1, 2 and 3.
\\
ICeCap -N -l sampleid \\
-s 1 -e 3 -G 0 -Z 1 -Q path/to/mappability/files -B path/to/baits.bed -P path/to/bowtiereference/files \\

Mandatory fields are the -N flag, the reference fasta file directory (-P), the sample id (-l), that should correspond to those used in the STAGE 1 above, and the (-f) path to the ICeCap folder.
If no baits BED file is specified (with the -B option), the pipeline will run in non-capture mode and allocate all fragment pairs to memory.
Note that the mappability files should be in the ICeCap folder, in a folder called $reference\_map$ or should be specified by thwe $-Q$ flag.
Setting -G 0 indicates that a restriction based (non-uniform) binning will be used.
Alternatively, if the -G option is set to 1, a uniform binning is chosen, with a size of r*Z bp. (see the -r and -Z flags below).
The -Z parameter will corse grain the analysis. So, for example, -G 0 -Z 1 corresponds to single fragment binning.
Similarly, -G 0 -Z 4 will correspond to a "meta-fragment" based binning, where four consecutive fragments are combined.
The -Z option can also be used with uniform bins, e.g. -G 1 -Z 5 will correspond to a uniform binning grid of mesh size 5*r=5kb. (default value 
of the -r option is $1000~bp$).
Interactions passing the FDR threshold, together with the associated q-values can be found in the "sampleid/stats/fdr" directory for further analysis and for plotting purposes. A Wash-U formatted file with all significant interactions will also be generated in the output folder, as specified by the -O option. Per fragment iWash files are grouped in chromosome start/end subfolder(s) for individual locus analysis.

\section{ICeCap Options:}
*******************\\
The following options can be passed to the ICeCap program.

************************************************************************************\\
**GENERAL OPTIONAL ARGUMENTS**

        -h, --help    Print this manual and exit.

        -V, --version   Print the version number and exit.

        -B, --baits path to file path to a BED file [without header lines] listing fragments that were enriched in the capture Hi-C experimental protocol.  If this option is not supplied, analysis is carried out on all fragments as in non-capture Hi-C. \\

        -b Specifies if Ligation is blunt. Default for CaptureC, specify e.g. -b hic if biotin filled non blunt end ligations occur in the experimental protocol
************************************************************************************ \\
**MANDATORY ARGUMENTS FOR STAGE 1**
 
        -E, --site-sequence nucleotide sequence string    Specifies the enzyme recognition site sequence.  Currently, only enzymes that cut at palindromic sequences are supported.
                                                            The default value is AAGCTT, which corresponds to the HindIII recognition seuqence.

        -C, --cut-position integer    Number of bases from the start of the recognition site sequence to the cut point.
                                                      
                                                     
 (e.g for HindIII use -C 1)

                     -AAGCTT-    -A       AGCTT-
                     -TTGCAA-    -TTGCA       A- \\
                                                      
                                                      
  (e.g. for MboI use -C 0)

                                                    
                      -GATC-      -       GATC-
                      -CTAG-      -CTAG       -  \\

        -L, --input-fastq prefix string    Path to the first set of fastq files . Sequencing read data contained in these files will be aligned.  Alignment will be carried out on upaired reads.  
                                           Reads will subsequently be paired according to matching read names.

        -R, --input-fastq prefix string    Path to the first set of fastq files . Sequencing read data contained in these files will be aligned.  Alignment will be carried out on upaired reads.  
                                           Reads will subsequently be paired according to matching read names.

        -P, --ref-bowtie prefix string     Path to the directory containing the Bowtie indices and reference fasta files. See the botwie2 manual for details on how to create these.

        -H  --prefixtobowtie         Default is Human reference Genome,hg19. Change to other reference build and/or species. 
        
        -T, --read-length integer    Length of the reads (in bases) found in the input fastq files. This option is superseeded as trimming of the flashed/non-flashed reads occur at the junction point rather than at a uniform read length. Leave default options.

        -Q, --refmapability prefix string    Path to the directory containing the files with mappability values [BED format].  The standard prefix is the one provided in the sample mappability folder provided with ICeCap. 

        -f, --dir-icecap path to directory     Path to the ICeCap (tools) directory containing the "referenceC" sub-directory.  The "referenceC" subdirectory contains scripts and templates used by ICeCap at various stages.  The default is the untar ICeCap directory.
       
        -l Sample Name/ID of the Biological replicate.\\

*******************************************************************\\
**OPTIONAL ARGUMENTS FOR STAGE 1**

        -D, --output-sam folder: The output directory where the files listing ditags in SAM format, as well as intermediate processing files (in the subdirectory "BOWTIE"), will be written.  This may include a prefix corresponding to the beginning of the file names.  The default is the current working directory.
       
        -A   --allele-specific-HiC. If yes, the CIGAR string is read and used to calibrate positions of the reverse reads before the matefixing step.
                                    This flag is experimental. Please use with care. Only Insertions/Deletions are considered for recalibration in the 
                                    present version. Leave default settings.

        -J  Number of cores used by Bowtie2.

******************************************************************\\
**MANDATORY ARGUMENTS FOR STAGE 2**

        -N, this flag will point to STAGE 2. It does not require an argument.
        -P, --ref-bowtie prefix string     Path to the directory containing the Bowtie indices and reference fasta files. See the botwie2 manual for details on how to create these.

        -l Sample Name/ID, if specified in STAGE 1.\\

        -f, --dir-icecap path to directory     Path to the ICeCap (tools) directory containing the "reference\_C" sub-directory.  The "reference\_C" subdirectory contains scripts and templates used by ICeCap at various stages.  The default is the untar ICeCap directory.

        -I Please provide a directory path here. IT should match the -D path, as given on stage 1, and corresponds to the folder where the Sample Name/ID folder and data subfolder containing the bona fide sam files generated in stage 1 are located. 

*****************************************************************\\
**OPTIONAL ARGUMENTS FOR STAGE 2**

      -o  Output prefix of choice to mark a specific instance of ICeCAP.
      -G, --uniform-bin-size integer    If this option is set, the interaction matrix will be constructed using bins of a uniform size in r bases (bp) as specified by the integer passed with the -r option.  Note that the RAM usage increases as a square of the number of bins so setting a low value here may consume a lot of RAM.  If this option is not set, the interaction matrix will correspond to fragments produced by the digestion of the genome with the chosen enzyme.

      -Z, --combine-intervals integer    If this option is set, fragments or bins will be combined into larger intervals for the purpose of calculating interactions.  The number of fragments or bins to be combined is indicated by the argument passed to this option.


      -S, --distribution string    Carry out statistical analysis on read count matrices.  Leave as default. Currently only pkY analysis occurs.

      -I, --input-sam folder: You can provide here the folder where you have generated the output files of STAGE 1. This flag, when used, should match the path specified in STAGE 1 by the        -D flag.

      -O, --output-bed prefix string    The output directory where the IBED file listing statistically significant interactions. The default is the ICeCap/SampleId directory.

      -s, --lowest-chromosome-number integer    Analysis will be implemented for chromsomes numbering between the values set by --lowest-chromosome-number and --highest-chromosome-number, inclusively.  The default value is 1. Please leave genome wide (default) settings.

      -e, --highest-chromosome-number integer    Analysis will be implemented for chromsomes numbering between the values set by --lowest-chromosome-number and --highest-chromosome-number, inclusively.  The default value is 24 . As above.

      -r  --minimum resolution integer. Number of base pairs. Default is 1000bp=1kb.

      -c, --weight-threshold float   This flag is currently superseeded. Iterative correction should be done using cool tools on .cool files.

      -m, --map-threshold float    Fragments/bins with an average mappability value below this threshold will be discarded from further analysis. The default value is 0.20.

      -t, --tolerance float    This flag is currently superseeded. Iterative correction should be done using cool tools on .cool files.

      -M, --max-iterations integer    Maximum number of iterations allowed for the Iterative Correction. As per -m/-t options.

      -W, --max-ditag-length integer    Maximum valid ditag length in bases (bp).  Ditags longer than this will be discarded. Default value is 800 bp.

      -U, --q-threshold   Leave as default. Currently only pkY analysis occurs.

      -p, --p-threshold   Leave as default. Currently only pkY analysis occurs.

      -w, --window  Currently pkY analysis occurs on a window size specified by this flag, in bp around each bait.

      -X, --dev Leave as default. Currently only pkY analysis occurs.

**************************************************************************\\
\section{Graphical Interface}

Please make sure to have the X11 libraries running on your desktop environment. 
At compile time, an executable named 'ICeCap\_plot' is generated. Please use the command line help option.
\section*{Updates}

Version 1.0-1.4 also includes graphical interface, used to plot heatmaps, topologically associating domains, significant interactions and Ensembl Genes. It is under current development and is based on the ANSI-C X11 graphical interface.

\end{document}
